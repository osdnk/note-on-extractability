% !TEX root = main.tex

\section{Introduction}
In this write-up, we focus on a ``regular'' folding protocol for the linear SIS-like relation. Particularly, we consider two instances of the linear relation, i.e. 
\[ 
    \mat{F} \mat{w}_0 = \mat{y}_0 \bmod q
    \quad \text{and} \quad
    \mat{F} \mat{w}_1 = \mat{y}_1 \bmod q,
\]
where $\inftynorm{\vec{w}_0}, \inftynorm{\vec{w}_1} \leq \beta$ and $q$ is some prime. Now, the folding protocol $\pifold$ is defined as follows:
\begin{enumerate}
    \item The verifier samples a (short) challenge ${c} \in \ring_q$ and sends it to the prover. 
    \item The prover computes $\mat{y}' = \mat{y}_0 + \mat{y}_1 \cdot c \bmod q$ and $\mat{w}' = \mat{w}_0 + \mat{w}_1 \cdot c \bmod q$.
    \item The verifier computes $\mat{w}' = \mat{w}_0 + \mat{w}_1 \cdot c \bmod q$.
\end{enumerate}

At the end of the protocol, both parties agree on the following relation: \[ \mat{F} \mat{w}' = \mat{y}' \bmod q. \]

In the context of the folding protocol and folding-based SNARKs, we are interested in extractability, i.e., we assume that we have oracle access to a (potentially cheating) prover $\prover^*$ outputting pairs $(\vec{w}', \vec{y}')$, and we want to extract a witness $\widehat{\mat{w}}$ such that $\mat{F} \widehat{\mat{w}} = \mat{y} \bmod q$. 

The routine strategy to construct such an extractor is to simply run the prover until obtaining two accepting transcripts for different challenges $c_0$ and $c_1$ under the 2-special-soundness framework. Then, the extractor has access to tuples $(c_0, \vec{w}_0')$ and $(c_1, \vec{w}_1')$ such that \[
    \mat{F} \vec{w}_0' = \vec{y}_0 + \vec{y}_1 \cdot c_0 \bmod q
    \quad \text{and} \quad
    \mat{F} \vec{w}_1' = \vec{y}_0 + \vec{y}_1 \cdot c_1 \bmod q.
\]

Then, by subtracting the two equations, we have \[
    \mat{F} (\vec{w}_0' - \vec{w}_1') = (\vec{y}_0 + \vec{y}_1 \cdot c_0) - (\vec{y}_0 + \vec{y}_1 \cdot c_1) \bmod q
    \quad \Rightarrow \quad
    \mat{F} (\vec{w}_0' - \vec{w}_1') / (c_0 - c_1) = \vec{y}_1 \bmod q.
\]

After renaming $\widehat{\vec{w}}_1 = \vec{w}_0' - \vec{w}_1'$ and $\hat{c}_1 = c_0 - c_1$, we have $\mat{F} \widehat{\vec{w}}_1 = \vec{y}_1 \cdot \hat{c}_1 \bmod q$. Similarly, we extract $\widehat{\vec{w}}_0$ and $\hat{c}_0$. We denote $\hat{c}_0$ and $\hat{c}_1$ as the slack factors.

Assuming that $\mat{F}$ is a (hard) SIS instance, then the extracted relation is still cryptographically meaningful, i.e., it is binding as after obtaining two preimages with slacks $(\widehat{\vec{w}}_0, \hat{c}_0)$ and $(\widehat{\vec{w}}_1, \hat{c}_1)$, we have 
\[ 
    \mat{F} (\widehat{\vec{w}}_1 \hat{c}_0 - \widehat{\vec{w}}_0 \hat{c}_1) = \vec{0} \bmod q,
\] 

i.e., $(\widehat{\vec{w}}_1 \hat{c}_0 - \widehat{\vec{w}}_0 \hat{c}_1)$ is an SIS-break.

However, the problem is that if we want to combine the protocol for many rounds, we get this annoying issue that the slack factor continuously grows in the extraction direction. Also, while extracting, we technically get a witness for a different relation (precisely, a relaxed variant of the input relation). For that reason, in the literature, several approaches have been proposed to reduce the slack factor, e.g.,

\begin{itemize}
    \item Usage of the subtractive set \cite{C:AlbLai21,C:CinLaiMal23,EPRINT:BonChe24}, where the challenge set is selected in such a way that the inverse of $\hat{c}$ is short. Then, $\vec{v} \defeq \widehat{\vec{w}} / \hat{c}$ can be treated as a candidate witness (with no slack).
    \item Unconditional norm checks, where the prover is required to prove that the norm of $\widehat{\vec{w}}$ is small, e.g., \cite{RR,EPRINT:BonChe25}. 
\end{itemize}

Also, in some literature, both strategies have been connected, e.g., \cite{AC:KLNO24,CC}, where the norm-check requires that the witness is already without slack, and the norm-check is used to further reduce the norm of the witness. 


\section{On the impossibility of extractability of folding without slack}

The idea we have been discussing is that potentially a random combination of short challenges should approximately preserve the norm. Indeed, there has been a line of research considering this approach, e.g., \cite{EC:GenHalLyu22,C:BeuSei23}, where the authors consider a random challenge set $\mathcal{C}$ and show that the norm of the extracted witness is approximately preserved. Nevertheless, these were always different challenge sets used for extraction and the norm preservation statistical argument. 

I tried different ways to make the extraction possible (with no slack) by using a challenge set of short ring elements. However, it seems impossible by the following argument. 

Let $\ring = \ZZ[X] / \langle X^N + 1 \rangle$\footnote{The argument generalizes to the complete family of cyclotomics}, where $N$ is a power-of-two. Observe that there exists a principal ideal $\mathcal{I} \defeq \langle X + 1 \rangle$ inside the ring $\ring$. The norm of this ideal is $2$, i.e., for any element $x \in \ring$, we have that either $(X + 1) \mid x$ or $(X + 1) \mid (x+1)$\footnote{Check the following script for the empirical test \url{https://gist.github.com/osdnk/b88f849c62084977e9d21165d06999b1}}. Therefore, we may assume that for the prover, for a non-negligible advantage, it is enough to produce the output for only one coset of the ideal, as at least half of the elements are in one coset.

Then, suppose that the prover has a witness $\vec{v}$ for a relaxed variant of the relation, i.e., 

\[ \mat{F} \underbrace{\left( \frac{\vec{w}}{X + 1} \right)}_{\vec{v}}  = \vec{y} \bmod q ~~\text{or}~~ \mat{F} \underbrace{\left( \frac{\vec{w} + 1}{X + 1} \right)}_{\vec{v}} = \vec{y} \bmod q, \]

where $\vec{w}$ is short, but $\vec{v}$ might have a slack. However, after folding, i.e., multiplying with a challenge $c \in \ring$ (selected from the appropriate coset), we have that $\vec{v} \cdot c$ is also short, so effectively the prover started from the wrong relation (not short) and folded into short, effectively attacking our candidate construction. 


As a conclusion, it seems that the folding protocol cannot be extractable without slack with an unstructured challenge set. The ways to make it extractable is to use a challenge set that is structured in such a way that the inverse-slack factor is small, e.g., the subtractive set \cite{C:AlbLai21,C:CinLaiMal23,EPRINT:BonChe24} or the norm-check \cite{RR,EPRINT:BonChe25}.